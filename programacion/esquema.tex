\begin{longenum}
    \item \textbf{\textsc{Introducción a la tecnología web}} \ev1\ \ra1\ (est: 2025\==09\==16)
    \begin{longenum}
        \item Introducción al desarrollo web
        \begin{longenum}
            \item Conceptos básicos
            \begin{longenum}
                \item Navegadores y servidores web
                \item Agentes de usuario
                \item Web estática vs. dinámica
                \item Estructura vs. contenido
                \item Arquitectura multinivel
            \end{longenum}
            \item Ejemplos de aplicaciones web
            \begin{longenum}
                \item Redes sociales: Facebook, Twitter…
                \item Comercio electrónico: Amazon, eBay…
                \item Administración electrónica…
                \item Portales
                \item ERP, CRM
            \end{longenum}
            \item Tecnologías de desarrollo de aplicaciones web
            \begin{longenum}
                \item .NET
                \item Java
                \item Ruby/Rails
                \item Python/Django
                \item PHP
                \item El Kung-Fu de la Programación
                \begin{longenum}
                    \item Odoo
                    \item PrestaShop
                    \item Drupal
                    \item WordPress
                \end{longenum}
            \end{longenum}
        \end{longenum}
        \item Arquitectura cliente/servidor
        \item HTML 5 básico (recordatorio de primer curso)
        \item Protocolo HTTP
        \begin{longenum}
            \item URIs
            \begin{longenum}
                \item URL encoding
            \end{longenum}
            \item Peticiones (\textit{HTTP requests}) y respuestas (\textit{HTTP responses})
            \item Métodos: GET, POST
            \item Versiones
            \begin{longenum}
                \item HTTP/1.0
                \item HTTP/1.1
            \end{longenum}
            \item Cabeceras HTTP
            \item Códigos de estado
            \item Experimentos
            \begin{longenum}
                \item \texttt{telnet} (a un servidor)
                \item \texttt{netcat} (desde un navegador)
                \item \texttt{curl -i -XPOST “http://…” | pygmentize -l http}
                \item \texttt{http}
                \item Google Chrome Developer Tools
            \end{longenum}
            \item Envío de datos al servidor
            \begin{longenum}
                \item Mediante GET
                \item Mediante POST
                \item Formularios HTML
            \end{longenum}
        \end{longenum}
    \end{longenum}
    \item \textbf{\textsc{Conceptos básicos de PHP (I)}} \ev1\ \ra2\ (est: 2025\==09\==23)
    \begin{longenum}
        \item Introducción a PHP
        \begin{longenum}
            \item Página web de PHP
            \item Instalación de PHP
            \item Configuración básica con \texttt{php.ini}
            \begin{longenum}
                \item \texttt{error\_reporting = E\_ALL}
                \item \texttt{display\_errors = On}
                \item \texttt{display\_startup\_errors = On}
                \item \texttt{date.timezone = 'UTC'}
            \end{longenum}
            \item Módulos de extensión
            \item Documentación y búsqueda de información
            \item Modos de ejecución \opcional\
            \begin{longenum}
                \item Por lotes
                \item Interactiva
                \begin{longenum}
                    \item \texttt{php -a}
                    \item PsySH
                \end{longenum}
            \end{longenum}
        \end{longenum}
        \item Sintaxis básica
        \begin{longenum}
            \item Datos e instrucciones
            \item Sentencias y comandos
            \begin{longenum}
                \item Comando \texttt{echo}
            \end{longenum}
            \item Expresiones, operadores y funciones
        \end{longenum}
        \item Funcionamiento del intérprete
        \begin{longenum}
            \item Código embebido
            \begin{longenum}
                \item Etiquetas \texttt{<?php} y \texttt{?>}
                \item Etiqueta \texttt{<?=}
                \item Servidor web  interno
            \end{longenum}
            \item Modo dual de operación
            \item Comentarios
        \end{longenum}
        \item Variables
        \begin{longenum}
            \item Conceptos básicos
            \item Destrucción de variables
            \item Asignación compuesta
            \item Operadores de asignación por valor y por referencia
            \item Variables predefinidas
        \end{longenum}
        \item Tipos básicos de datos
        \begin{longenum}
            \item Lógicos (\texttt{bool})
            \begin{longenum}
                \item Operadores lógicos
            \end{longenum}
            \item Numéricos
            \begin{longenum}
                \item Enteros (\texttt{int})
                \item Números en coma flotante (\texttt{float})
                \item Operadores
                \begin{longenum}
                    \item Operadores aritméticos
                    \item Operadores de incremento/decremento
                \end{longenum}
            \end{longenum}
            \item Cadenas (\texttt{string})
            \begin{longenum}
                \item Operadores de cadenas
                \begin{longenum}
                    \item Concatenación
                    \item Acceso y modificación por caracteres
                    \item Operador de incremento \opcional\
                \end{longenum}
                \item Funciones de manejo de cadenas
                \item Extensión \textit{mbstring}
            \end{longenum}
            \item Nulo (\texttt{null})
        \end{longenum}
        \item Manipulación de datos
        \begin{longenum}
            \item Precedencia de operadores
            \item Operadores de asignación compuesta
            \item Comprobaciones
            \begin{longenum}
                \item De tipos
                \begin{longenum}
                    \item \texttt{gettype()}
                    \item \texttt{is\_*()}
                \end{longenum}
                \item De valores
                \begin{longenum}
                    \item \texttt{is\_numeric()}
                    \item \texttt{ctype\_*()}
                \end{longenum}
            \end{longenum}
            \item Conversiones de tipos
            \begin{longenum}
                \item Conversión explícita (\textit{casting}) vs. implícita (automática)
                \item Conversión a \texttt{bool}
                \item Conversión a \texttt{int}
                \item Conversión a \texttt{float}
                \item Conversión de \texttt{string} a número
                \item Conversión a \texttt{string}
                \item Funciones de obtención de valores
                \begin{longenum}
                    \item \texttt{intval()}
                    \item \texttt{floatval()}
                    \item \texttt{strval()}
                    \item \texttt{boolval()}
                \end{longenum}
                \item Funciones de formateado numérico
                \begin{longenum}
                    \item \texttt{number\_format()}
                    \item \texttt{money\_format()}
                    \begin{longenum}
                        \item \texttt{setlocale()}
                    \end{longenum}
                    \item \texttt{NumberFormatter}
                \end{longenum}
            \end{longenum}
            \item Comparaciones
            \begin{longenum}
                \item Operadores de comparación
                \item \texttt{==} vs. \texttt{===}
                \item Ternario (\texttt{?:})
                \item Fusión de \texttt{null} (\texttt{??})
                \item Reglas de comparación de tipos
            \end{longenum}
        \end{longenum}
        \item Constantes
        \begin{longenum}
            \item \texttt{define()} y \texttt{const}
            \item Constantes predefinidas
            \item \texttt{defined()}
        \end{longenum}
    \end{longenum}
    \item \textbf{\textsc{Conceptos básicos de PHP (II)}} \ev1\ \ra3\ (est: 2025\==09\==30)
    \begin{longenum}
        \item Estructuras de control
        \begin{longenum}
            \item Secuencia
            \item Selección
            \item Iteración
            \item Sintaxis alternativa
        \end{longenum}
        \item Funciones predefinidas destacadas
        \begin{longenum}
            \item \texttt{isset()}
            \item \texttt{empty()}
            \item \texttt{var\_dump()}
        \end{longenum}
        \item Arrays
        \begin{longenum}
            \item Operadores para arrays
            \begin{longenum}
                \item Acceso, modificación y agregación
            \end{longenum}
            \item Funciones de manejo de arrays
            \begin{longenum}
                \item Ordenación de arrays
                \item \texttt{print\_r()}
                \item \texttt{'+'} vs. \texttt{array\_merge()}
                \item \texttt{isset()} vs. \texttt{array\_key\_exists()}
            \end{longenum}
            \item \texttt{foreach}
            \item Conversión a \texttt{array}
            \item \textit{Ejemplo}: \texttt{\$argv} en CLI
            \item Funciones auxiliares de interés
            \begin{longenum}
                \item \texttt{extract()}
                \item \texttt{compact()}
            \end{longenum}
        \end{longenum}
        \item Formularios web
        \begin{longenum}
            \item Manejo de datos de entrada: \texttt{\$\_GET} y \texttt{\$\_POST}
            \item Tailwind CSS
        \end{longenum}
        \item Funciones definidas por el usuario
        \begin{longenum}
            \item Argumentos
            \begin{longenum}
                \item Paso de argumentos por valor y por referencia
                \item Argumentos por defecto
            \end{longenum}
            \item Ámbito de variables
            \begin{longenum}
                \item Ámbito simple al archivo
                \item Variables locales
                \item Uso de \texttt{global}
                \item Variables superglobales
            \end{longenum}
            \item Declaraciones de tipos
            \begin{longenum}
                \item Declaraciones de tipo de argumento
                \item Declaraciones de tipo de devolución
                \item Tipos \textit{nullable} (\texttt{?}) y \texttt{void}
                \item Tipificación estricta
            \end{longenum}
            \item Funciones anónimas
            \begin{longenum}
                \item Clausuras
            \end{longenum}
            \item Callables
            \begin{longenum}
                \item \texttt{call\_user\_func()}
                \item \texttt{array\_map()} y \texttt{array\_reduce()}
            \end{longenum}
        \end{longenum}
        \item Comentarios y documentación del código
    \end{longenum}
    \item \textbf{\textsc{Aspectos básicos de orientación a objetos}} \ra5\ (est: 2025\==10\==07)
    \begin{longenum}
        \item Objetos
        \begin{longenum}
            \item \texttt{new}
            \item \texttt{instanceof}
            \item \texttt{get\_class}
        \end{longenum}
        \item Referencias
        \begin{longenum}
            \item Asignación por referencia (\texttt{=\&})
        \end{longenum}
        \item Clonación de objetos
        \item Comparación de objetos
        \item Propiedades
        \begin{longenum}
            \item Predeterminadas
            \item Dinámicas
        \end{longenum}
        \item Métodos
        \item Constantes
        \begin{longenum}
            \item Operador de resolución de ámbito (\texttt{::})
        \end{longenum}
        \item \textit{Ejemplo}: manejo de fechas, horas, instantes e intervalos
        \item Excepciones
        \begin{longenum}
            \item Manejo de errores clásico en PHP
            \item Errores vs. excepciones
            \item La clase \texttt{Exception}
            \item La clase \texttt{Error}
            \item La clase \texttt{ErrorException}
            \item Estructura de control \texttt{try ... catch}
        \end{longenum}
    \end{longenum}
    \item \textbf{\textsc{Persistencia de datos con PHP}} \ev1\ \ra6\ (est: 2025\==10\==14)
    \begin{longenum}
        \item PDO (PHP Data Objects)
        \begin{longenum}
            \item Clase \texttt{PDO}
            \begin{longenum}
                \item \texttt{\_\_construct()}
                \item \texttt{query()}
                \item \texttt{exec()}
                \item \texttt{prepare()}
            \end{longenum}
            \item Clase \texttt{PDOStatement}
            \begin{longenum}
                \item \texttt{fetch()}
                \item \texttt{fetchAll()}
                \item \texttt{fetchColumn()}
                \item \texttt{execute()}
                \item \texttt{rowCount()}
            \end{longenum}
            \item Correspondencias de tipos entre SQL y PHP
            \item Manejo de errores en PDO
            \begin{longenum}
                \item \texttt{PDO::errorInfo()}
            \end{longenum}
            \item Transacciones
            \begin{longenum}
                \item \texttt{\$pdo->beginTransaction();}
                \item \texttt{\$pdo->commit();}
                \item \texttt{\$pdo->rollBack();}
            \end{longenum}
        \end{longenum}
        \item Programación de \textit{CRUD} básico
        \begin{longenum}
            \item Ejemplo de aplicación: \textit{Muéveme} 
            \item Ejemplo de aplicación: \textit{FilmAffinity} 
        \end{longenum}
        \item Post/Redirect/Get
        \item \texttt{header()}
        \begin{longenum}
            \item \texttt{output\_buffering}
        \end{longenum}
    \end{longenum}
    \item \textbf{\textsc{Desarrollo de aplicaciones con PHP}} \ev1\ \ra4\ (est: 2025\==10\==21)
    \begin{longenum}
        \item Cookies
        \begin{longenum}
            \item \texttt{setcookie()}
            \item Ejemplos de uso
        \end{longenum}
        \item Sesiones
        \begin{longenum}
            \item Iniciar una sesión
            \begin{longenum}
                \item \texttt{session\_start()}
            \end{longenum}
            \item Usar una sesión
            \begin{longenum}
                \item \texttt{\$\_SESSION}
                \item Ejemplos de uso
            \end{longenum}
            \item Terminar una sesión
            \begin{longenum}
                \item \texttt{session\_destroy()}
                \item \texttt{session\_name()}
                \item \texttt{session\_id()}
                \item \texttt{session\_get\_cookie\_params()}
            \end{longenum}
        \end{longenum}
        \item Autenticación
        \begin{longenum}
            \item Contraseñas
            \begin{longenum}
                \item https://www.md5online.org/
                \item https://www.sha1online.org/
                \item \texttt{password\_hash()}
                \item \texttt{password\_verify()}
            \end{longenum}
        \end{longenum}
        \item Seguridad
        \begin{longenum}
            \item Filtrado y formateado
            \begin{longenum}
                \item Filtrado de la entrada
                \begin{longenum}
                    \item Cómo \textit{NO} se debe hacer
                    \item Extensión Filter
                    \begin{longenum}
                        \item \texttt{filter\_input()}, \texttt{filter\_has\_var()}, \texttt{filter\_var()}
                        \item Filtros de validación y saneado
                    \end{longenum}
                    \item Expresiones regulares (PCRE)
                \end{longenum}
                \item Formateado de datos
                \begin{longenum}
                    \item Números y cantidades monetarias
                    \item Fechas
                \end{longenum}
            \end{longenum}
            \item Seguridad básica
            \begin{longenum}
                \item Filtrar la entrada, escapar la salida
                \item Cross-Site Scripting (XSS)
                \begin{longenum}
                    \item No persistente
                    \item Persistente
                    \item Escapado de la salida
                    \begin{longenum}
                        \item HTML Purifier
                    \end{longenum}
                \end{longenum}
                \item \texttt{htmlspecialchars()}
            \end{longenum}
            \item Inyección de código SQL
            \item Cross-Site Request Forgery (CSRF)
        \end{longenum}
        \item Depuración
        \begin{longenum}
            \item \texttt{var\_dump()}, \texttt{print\_r()}, \texttt{die()}
            \item Depuración con PsySH
            \item Xdebug \opcional\
            \begin{longenum}
                \item Módulo Xdebug
                \item Aplicación Xdebug para Chrome
                \item Extensión Xdebug Helper para Chrome
                \item Paquete \texttt{php-debug} para Atom
            \end{longenum}
        \end{longenum}
    \end{longenum}
    \item \textbf{\textsc{Diseño de aplicaciones orientadas a objetos}} \ev1\ \ra5\ (est: 2025\==10\==28)
    \begin{longenum}
        \item Clases
        \item Propiedades
        \begin{longenum}
            \item Predeterminadas
            \item Dinámicas
        \end{longenum}
        \item Métodos
        \begin{longenum}
            \item Referencia \texttt{\$this}
            \item Constructores y destructores
        \end{longenum}
        \item Constantes
        \begin{longenum}
            \item \texttt{self}
        \end{longenum}
        \item Herencia
        \begin{longenum}
            \item \texttt{parent}
            \item Sobreescritura de métodos
        \end{longenum}
        \item Miembros estáticos
        \begin{longenum}
            \item Constantes
            \item Métodos estáticos
            \item Propiedades estáticas
            \item Enlace estático en tiempo de ejecución
        \end{longenum}
        \item Interfaces
        \item Traits
        \item Contravarianza y covarianza
        \item La clase \texttt{stdClass}
        \begin{longenum}
            \item Conversión de \texttt{array} a \texttt{object} y viceversa.
        \end{longenum}
    \end{longenum}
    \item \textbf{\textsc{Interoperabilidad}} \ev1\ \ra9\ (est: 2025\==11\==04)
    \begin{longenum}
        \item Programación modular en PHP
        \begin{longenum}
            \item Inclusión de scripts
            \begin{longenum}
                \item \texttt{include}, \texttt{require}
                \item \texttt{include\_once}, \texttt{require\_once}
            \end{longenum}
            \item Espacios de nombres
            \item Autocarga de clases
            \begin{longenum}
                \item \texttt{spl\_autoload\_register()}
                \item PSR-4
            \end{longenum}
        \end{longenum}
        \item Composer
        \begin{longenum}
            \item Versionado semántico
            \item Paquetes
            \item Packagist
            \item Dependencias
            \begin{longenum}
                \item \texttt{composer.json} y \texttt{composer.lock}
            \end{longenum}
            \item Versiones y restricciones
            \begin{longenum}
                \item Versión exacta
                \item Rango (\texttt{>}, \texttt{>=}, \texttt{<}, \texttt{<=}, \texttt{!=}, \texttt{ }, \texttt{,}, \texttt{||})
                \item Guión (\texttt{-})
                \item Asterisco (\texttt{*})
                \item Tilde (\texttt{~})
                \item Circunflejo (\texttt{\^})
                \item Nombres de rama
                \begin{longenum}
                    \item \texttt{dev-master}
                    \item \texttt{5.1.x-dev}
                \end{longenum}
                \item Estabilidad mínima
                \item Comprobador online de restricciones
            \end{longenum}
            \item Comandos básicos
            \begin{longenum}
                \item \texttt{require}
                \item \texttt{install}
                \item \texttt{update}
                \item \texttt{dump-autoload}
            \end{longenum}
            \item Entornos de desarrollo y producción
            \item Autoloader de Composer
        \end{longenum}
        \item Consumo de API de terceros
        \begin{longenum}
            \item \texttt{guzzlehttp/guzzle}
        \end{longenum}
        \item Recomendaciones PSR del PHP-FIG (Framework Interop Group) \opcional\
        \begin{longenum}
            \item PSR-1: Basic Coding Standard
            \item PSR-2: Coding Style Guide
            \item PSR-4: Autoloading Standard
            \item PSR-5: PHPDoc Standard (borrador)
            \item PSR-11: Extended Coding Style Guide (borrador)
            \item PSR-19: PHPDoc tags (borrador)
            \item Herramientas externas \opcional\
            \begin{longenum}
                \item PHP\_CodeSniffer
                \item PHP-CS-Fixer
            \end{longenum}
        \end{longenum}
    \end{longenum}
    \item \textbf{\textsc{Introducción a Laravel}} \ev1\ \ra5\ (est: 2025\==11\==11)
    \begin{longenum}
        \item Frameworks, microframeworks y librerías
        \item Patrón Modelo-Vista-Controlador (MVC)
        \begin{longenum}
            \item Modelos
            \item Vistas
            \item Controladores
            \item Rutas
        \end{longenum}
        \item Laravel
        \begin{longenum}
            \item ¿Qué es Laravel
            \item ¿Cómo es Laravel comparado con otros frameworks?
            \item Versiones de Laravel
        \end{longenum}
        \item Instalación, requisitos y puesta en marcha
        \begin{longenum}
            \item Requisitos previos
            \item Instalación de Laravel
        \end{longenum}
        \item ¡Hola, mundo!
        \item Herramientas de desarrollo
        \begin{longenum}
            \item Clockwork
            \item Trazas de depuración
        \end{longenum}
        \item Estilo del código
        \item Arquitectura de Laravel
        \begin{longenum}
            \item El ciclo de vida de una petición
            \item El contenedor de servicios
            \item Proveedores de servicios
            \item Fachadas
        \end{longenum}
        \item Gestión de peticiones en Laravel
        \begin{longenum}
            \item Enrutado
            \item Middlewares
            \item Peticiones
            \item Respuestas
            \item Controladores
            \item Sesiones y cookies
            \item Gestión de errores
        \end{longenum}
    \end{longenum}
    \item \textbf{\textsc{Visualización de datos en Laravel}} \ev1\ \ra5\ (est: 2025\==11\==18)
    \begin{longenum}
        \item Vistas
        \item Plantillas Blade
        \item Daisy UI
        \item Compilación de assets
        \item Generación de URLs
        \item Creación y validación de formularios en Laravel
        \begin{longenum}
            \item Creación de formularios
            \item Protección contra CSRF
            \item Validación de la entrada
            \item Subida de archivos
        \end{longenum}
    \end{longenum}
    \item \textbf{\textsc{Bases de datos en Laravel}} \ev1\ \ra6\ (est: 2025\==11\==25)
    \begin{longenum}
        \item Conceptos básicos
        \item Query Builder
        \item Migraciones
        \item Paginación
        \item Semillas
        \item Mapeado objeto-relacional con Eloquent
        \begin{longenum}
            \item Conceptos básicos
            \item Relaciones
            \begin{longenum}
                \item Uno a uno
                \item Uno a muchos
                \item Muchos a muchos
                \item Relaciones polimórficas
                \item Carga ansiosa
                \begin{longenum}
                    \item El problema del \textit{N + 1}
                \end{longenum}
            \end{longenum}
            \item Colecciones
            \item Mutadores y castings
        \end{longenum}
    \end{longenum}
    \item \textbf{\textsc{Seguridad y autenticación en Laravel}} \ev2\ \ra4\ (est: 2026\==01\==07)
    \begin{longenum}
        \item Cookies
        \item Sesiones
        \item Autenticación
        \begin{longenum}
            \item Starter Kits
            \item Laravel Breeze
        \end{longenum}
        \item Autorización
        \item Verificación de email
        \item Cifrado
        \item Hashing
        \item Reinicio de contraseñas
    \end{longenum}
    \item \textbf{\textsc{Tecnologías asíncronas servidor-cliente}} \ev2\ \ra8\ (est: 2026\==01\==14)
    \begin{longenum}
        \item Livewire
        \begin{longenum}
            \item Alpine.js
            \item Mary UI
        \end{longenum}
        \item Filament PHP
        \item Inertia
    \end{longenum}
    \item \textbf{\textsc{Servicios web con Laravel}} \ev2\ \ra7\ (est: 2026\==01\==21)
    \begin{longenum}
        \item Recursos API
        \item Serialización
        \item Limitación de frecuencia de peticiones
        \item Documentación de la API
    \end{longenum}
    \item \textbf{\textsc{Características adicionales de Laravel}} \ev3\ \opcional\ (est: 2026\==05\==04)
    \begin{longenum}
        \item Correo electrónico
        \item Helpers
        \item Eventos
        \item Contratos
        \item Colecciones
        \item Colas
        \item Planificación de tareas
        \item Almacenamiento de archivos
    \end{longenum}
    \item \textbf{\textsc{Calidad}} \ev3\ \opcional\ (est: 2026\==05\==11)
    \begin{longenum}
        \item Pruebas
        \begin{longenum}
            \item Pruebas de HTTP
            \item Pruebas de consola
            \item Pruebas de navegador
            \item Bases de datos
            \item Mocking
        \end{longenum}
        \item Análisis estático
        \begin{longenum}
            \item Larastan
        \end{longenum}
        \item Depuración
        \begin{longenum}
            \item dd()
            \item ddd()
            \item Clockwork
        \end{longenum}
        \item Documentación \opcional\
        \begin{longenum}
            \item phpDocumentor
            \item GitHub Pages
        \end{longenum}
        \item Mantenimiento y calidad del código
        \begin{longenum}
            \item CodeSniffer
            \item CS\_Fixer
            \begin{longenum}
                \item Laravel Pint
            \end{longenum}
        \end{longenum}
    \end{longenum}
\end{longenum}
