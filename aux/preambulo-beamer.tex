\usepackage{etoolbox}

% Por defecto, muestra el texto centrado en las diapositivas:
\makeatletter
\apptocmd\beamer@framenotesbegin
  {\centering}
  {}{}
\makeatother
\AtBeginDocument{%
  \letcs\oig{@orig\string\includegraphics}%
  \renewcommand<>\includegraphics[2][]{%
    \only#3{%
      {\centering\oig[{#1}]{#2}\par}%
    }%
  }%
}

% El texto monoespaciado se muestra de otro color:
\definecolor{ttcolor}{RGB}{64,112,161}
\let\Oldtexttt\texttt
\renewcommand\texttt[1]{{\ttfamily\color{ttcolor}#1}}

% Cambia varios tamaños de tipografías:
\setbeamersize{text margin left=4em,text margin right=4em}
\setbeamerfont{section title}{size={\fontsize{16}{18}}}
\setbeamerfont{title}{size={\fontsize{18}{26}}}

% Los bloques de código se muestran enmarcados:
\usepackage[framemethod=TikZ]{mdframed}
%\usepackage{xpatch}
%\makeatletter
%\xpatchcmd{\endmdframed}
%  {\aftergroup\endmdf@trivlist\color@endgroup}
%  {\endmdf@trivlist\color@endgroup\@doendpe}
%  {}{}
%\makeatother
\mdfdefinestyle{codedefault}{%
backgroundcolor=black!4,
roundcorner=2pt,
linecolor=black!30,
linewidth=0.5pt,
leftmargin=0,
rightmargin=0,
innerleftmargin=5,
innerrightmargin=5,
skipabove=5,
skipbelow=5}
% redefining the Shaded-Environment of the pandoc-template to allow customised
% style-settings
\ifdefined\Shaded
  \renewenvironment{Shaded}{\centering \begin{mdframed}[style=codedefault]}{\end{mdframed}}
\fi

% Adjust fontsize for code-blocks if necessary (by redefining the
% Highlighting-Environment of the pandoc-template)
\ifdefined\Highlighting
  \DefineVerbatimEnvironment{Highlighting}{Verbatim}{commandchars=\\\{\},fontsize=\footnotesize}
\fi

\makeatletter
\def\verbatim@nolig@list{}
\makeatother

% Apaño para que Pandoc traduzca lo que hay entre \begin{...} y \end{...}:
\newcommand{\hideFromPandoc}[1]{#1}
\hideFromPandoc{
    \let\Begin\begin
    \let\End\end
}

\usepackage[shortlabels]{enumitem}
\setlist[enumerate]{label*=\arabic*.,leftmargin=*}
\setlist[itemize]{label=\usebeamerfont*{itemize item}\usebeamercolor[fg]{itemize item}\usebeamertemplate{itemize item}}

\usepackage[answerdelayed]{exercise}
\renewcommand{\ExerciseName}{Pregunta}
\renewcommand{\AnswerName}{Respuesta}
\renewcommand{\ExerciseHeader}{\textbf{\large%
\ExerciseName\ \ExerciseHeaderNB\ExerciseHeaderTitle%
\ExerciseHeaderOrigin\medskip}\par}
\renewcommand{\AnswerHeader}{\textbf{%
Respuesta a la \ExerciseName\ \ExerciseHeaderNB}\smallskip\par}
