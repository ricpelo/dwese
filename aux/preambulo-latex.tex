\usepackage{etoolbox}

% El texto monoespaciado se muestra de otro color:
\definecolor{ttcolor}{RGB}{64,112,161}
\let\Oldtexttt\texttt
\renewcommand\texttt[1]{{\ttfamily\color{ttcolor}#1}}

% Los bloques de código se muestran enmarcados:
\usepackage[xparse,skins,breakable]{tcolorbox}
\DeclareTColorBox{caja}{}
    { center,width=1\linewidth,enhanced,fonttitle=\large\bfseries,
      drop fuzzy shadow,breakable,
      boxrule=0.2mm,
      colback=black!4!white,
      before skip=0.5cm,after skip=0.5cm,
      top=0.1cm,
      bottom=0.1cm,
      left=0.1cm,
      right=0.1cm
    }

\ifdefined\Shaded
  \renewenvironment{Shaded}{\centering \begin{caja}}{\end{caja}}
\fi

% Adjust fontsize for code-blocks if necessary (by redefining the
% Highlighting-Environment of the pandoc-template)
\ifdefined\Highlighting
  \DefineVerbatimEnvironment{Highlighting}{Verbatim}{commandchars=\\\{\},fontsize=\footnotesize}
\fi

\makeatletter
\def\verbatim@nolig@list{}
\makeatother

% En los apuntes no hay columnas al estilo Beamer:
\newenvironment{columns}[1][1]{}{}
\newenvironment{column}[1]{}{}

% Apaño para que Pandoc traduzca lo que hay entre \begin{...} y \end{...}:
\newcommand{\hideFromPandoc}[1]{#1}
\hideFromPandoc{
    \let\Begin\begin
    \let\End\end
}

% Puntos tras los números de sección y en la tabla de contenidos:
\usepackage{titlesec}
\usepackage[dotinlabels]{titletoc}
\titlelabel{\thetitle.\quad}

\makeatletter
\def\fps@figure{h}
\makeatother

% No muestra "Figura #: ":
\usepackage{caption}
\captionsetup[figure]{labelformat=empty}

% Se puede hacer un salto de línea tras :: en texto monoespaciado:
\usepackage{xparse}
\ExplSyntaxOn
\NewDocumentCommand{\replace}{mmm}
 {
  \marian_replace:nnn {#1} {#2} {#3}
 }
\tl_new:N \l_marian_input_text_tl
\cs_new_protected:Npn \marian_replace:nnn #1 #2 #3
 {
  \tl_set:Nn \l_marian_input_text_tl { #1 }
  \tl_replace_all:Nnn \l_marian_input_text_tl { #2 } { #3 }
  \tl_use:N \l_marian_input_text_tl
 }
\ExplSyntaxOff
\usepackage{url}
\let\OldTexttt\texttt
\renewcommand{\texttt}[1]{\OldTexttt{\replace{#1}{::}{::\allowbreak}}}

\usepackage[shortlabels]{enumitem}
\setlist[enumerate]{label*=\arabic*.,leftmargin=*}

\usepackage[answerdelayed]{exercise}
\renewcommand{\ExerciseName}{Pregunta}
\renewcommand{\AnswerName}{Respuesta}
\renewcommand{\ExerciseHeader}{\textbf{\large%
\ExerciseName\ \ExerciseHeaderNB\ExerciseHeaderTitle%
\ExerciseHeaderOrigin\medskip}\par}
\renewcommand{\AnswerHeader}{\textbf{%
Respuesta a la \ExerciseName\ \ExerciseHeaderNB}\smallskip\par}
\usepackage{rotating}

% Para que funcione la marca ✔:
\usepackage{newunicodechar}
\newfontface{\libertinus}{libertinusserif-regular.otf} % a font that has ✔
\newunicodechar{✔}{%
  \begingroup
  \normalfont\libertinus ✔%
  \endgroup
}
