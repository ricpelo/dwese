\usepackage{etoolbox}

% Los bloques de código se muestran enmarcados:
\usepackage[xparse,skins,breakable]{tcolorbox}

% El texto monoespaciado se muestra de otro color:
\definecolor{ttcolor}{RGB}{64,112,161}
\let\Oldtexttt\texttt
\renewcommand\texttt[1]{{\ttfamily\color{ttcolor}#1}}

\ifdefined\Shaded
  \renewenvironment{Shaded}{\centering \begin{caja}}{\end{caja}}
\fi

% Adjust fontsize for code-blocks if necessary (by redefining the
% Highlighting-Environment of the pandoc-template)
\ifdefined\Highlighting
  \DefineVerbatimEnvironment{Highlighting}{Verbatim}{commandchars=\\\{\},fontsize=\footnotesize}
\fi

% Activa ligaduras en verbatim:
\makeatletter
\def\verbatim@nolig@list{}
\makeatother

% Apaño para que Pandoc traduzca lo que hay entre \begin{...} y \end{...}:
\newcommand{\hideFromPandoc}[1]{#1}
\hideFromPandoc{
    \let\Begin\begin
    \let\End\end
}

% Las listas se numeran estilo 1.2.3.4:
\usepackage[shortlabels]{enumitem}
\setlist[enumerate]{label*=\arabic*.,leftmargin=*}

% Se puede hacer un salto de línea tras :: en texto monoespaciado
% (útil, por ejemplo, en yii\base\Component::__construct()):
\usepackage{xparse}
\ExplSyntaxOn
\NewDocumentCommand{\replace}{mmm}
 {
  \marian_replace:nnn {#1} {#2} {#3}
 }
\tl_new:N \l_marian_input_text_tl
\cs_new_protected:Npn \marian_replace:nnn #1 #2 #3
 {
  \tl_set:Nn \l_marian_input_text_tl { #1 }
  \tl_replace_all:Nnn \l_marian_input_text_tl { #2 } { #3 }
  \tl_use:N \l_marian_input_text_tl
 }
\ExplSyntaxOff
\usepackage{url}
\let\OldTexttt\texttt
\renewcommand{\texttt}[1]{\OldTexttt{\replace{#1}{::}{::\allowbreak}}}

% Preguntas y respuestas:
\usepackage[answerdelayed]{exercise}
\renewcommand{\ExerciseName}{Pregunta}
\renewcommand{\AnswerName}{Respuesta}
\renewcommand{\ExerciseHeader}{\textbf{\large%
\ExerciseName\ \ExerciseHeaderNB\ExerciseHeaderTitle%
\ExerciseHeaderOrigin\medskip}\par}
\renewcommand{\AnswerHeader}{\textbf{%
Respuesta a la \ExerciseName\ \ExerciseHeaderNB}\smallskip\par}

% Para que funcione la marca ✔:
\usepackage{newunicodechar}
\newfontface{\libertinus}{libertinusserif-regular.otf} % a font that has ✔
\newunicodechar{✔}{%
  \begingroup
  \normalfont\libertinus ✔%
  \endgroup
}

% Tipografías para matemáticas:
%\usepackage{unicode-math}
%\setmathsfont(Latin,Greek)[ItalicFont=LinBiolinum_RI.otf,
%    BoldFont=LinBiolinum_RB.otf]{LinBiolinum_R.otf}
\setmathsfont(Greek){xits-math.otf}
\setmathsfont(Latin,Digits){Lato}
%\setmathfont[range=\mathit]{Lato Italic}
